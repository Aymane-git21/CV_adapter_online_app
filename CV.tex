% We use 'standalone' to crop the page to the content height.
% varwidth=18.6cm calculates the text width (A4 21cm - 1.2cm left - 1.2cm right).
% border=1.2cm adds the margins back around the content.
\documentclass[10pt, varwidth=18.6cm, border=1.2cm]{standalone}

\usepackage[utf8]{inputenc}
\usepackage[T1]{fontenc}
\usepackage[french]{babel}

% Note: 'geometry' package is removed because 'standalone' handles the page size.
\usepackage{titlesec}
\usepackage{hyperref}
\usepackage{fontawesome5}
\usepackage{xcolor}

% --- Colors ---
\definecolor{darkblue}{RGB}{0,32,96}
\definecolor{linkcolor}{RGB}{0,80,160}

% --- General Settings ---
\setlength{\parindent}{0pt}
\setlength{\parskip}{4pt}

% --- Hyperlinks ---
\hypersetup{
    colorlinks=true,
    linkcolor=linkcolor,
    urlcolor=linkcolor,
}

% --- Section Formatting ---
\titleformat{\section}
  {\large\bfseries\color{darkblue}\uppercase}
  {}{0em}
  {}[\titlerule]
\titlespacing{\section}{0pt}{10pt}{5pt}

% --- Custom Commands ---
\newcommand{\entry}[4]{
  \vspace{2pt}
  \noindent
  \begin{tabular*}{\textwidth}{@{\extracolsep{\fill}}lr}
    \textbf{#1} & \textit{#2} \\
    \textit{#3} & \small #4 \\
  \end{tabular*}
  \vspace{4pt}
}

\newcommand{\project}[2]{
  \textbf{#1} \hfill \textit{\footnotesize #2} \par
  \vspace{2pt}
}

\begin{document}
% No \pagestyle{empty} needed, standalone does this automatically.
\pagestyle{empty}

% --- Header ---
\begin{center}
    {\Huge \textbf{Aymane Merbouh}} \\[4pt]
    {\Large Étudiant Master 2 | IA\&RA, Réalité Mixte (AR/VR)} \\[6pt]
    \small
    \faMapMarker* \ Valenciennes / Paris \quad
    \faPhone \ +33 6 60 68 67 20 \quad
    \faEnvelope \ \href{mailto:aymanemerbouh03@gmail.com}{aymanemerbouh03@gmail.com} \\
    \faLinkedin \ \href{https://linkedin.com/in/aymane-merbouh}{linkedin.com/in/aymane-merbouh} \quad
    \faGithub \ \href{https://github.com/Aymane-git21}{github.com/Aymane-git21}
\end{center}

% --- Profile ---
\section{Profil}
Étudiant en Master 2 Informatique spécialisé dans les systèmes du métavers. Passionné par l'IA, la VR et la recherche appliquée, je souhaite rejoindre une entreprise ou un laboratoire innovant pour contribuer à des projets technologiques à fort impact.

% --- Education ---
\section{Formation}

\entry{Master Informatique (Systèmes \& Technologies du Métavers)}{2024 -- Présent}{INSA / UPHF Valenciennes}{Valenciennes, France}
\textbf{Modules :} Réalité Virtuelle/Augmentée, Jumeaux Numériques, IA, Simulation, Big Data, Web.

\entry{BUT Génie Électrique et Informatique Industrielle (GEII)}{2021 -- 2024}{IUT de l'Aisne}{France}
\textbf{Compétences :} Informatique Industrielle (C/C++), Automatisme, Systèmes Embarqués.

% --- Experience ---
\section{Expérience Professionnelle}

\entry{Stagiaire Recherche IA \& VR}{Mars -- Août 2024}{UM6P School of Collective Intelligence}{Maroc}
Intégration de modules IA (NLP + analyse comportementale) pour l'étude d’interactions de groupe.  
Création de tableaux de bord VR temps réel pour visualiser des métriques d'intelligence collective.

\entry{Développeur Freelance}{2023 -- Présent}{Indépendant}{Télétravail}
Développement d’applications mobiles multiplateformes (Flutter, C\#).  
Analyse des besoins clients, prototypage et livraison d'applications fonctionnelles.

\entry{Stagiaire IT}{Avril -- Juin 2023}{E-cube Technology}{Soissons, France}
Amélioration d'un système industriel piloté par Raspberry Pi et Node-RED.  
Développement et maintenance d’interfaces web (HTML/CSS/JS).

% --- Skills ---
\section{Compétences Techniques}

\textbf{AR/VR \& Dév :} Unity, C\#, Logique de jeu, UI/UX XR, Blender (bases).   \\
\textbf{Langages \& Web :} Python, C++, Java, JavaScript (Node.js), Dart (Flutter), HTML/CSS.   \\
\textbf{Data \& IA :} PyTorch, OpenCV, Hadoop (MapReduce), Vision par ordinateur, Détection d’objets.  \\
\textbf{Outils \& DevOps :} Git/GitHub, Docker, Linux, Android Studio.

% --- Projects ---
\section{Projets Académiques}

\project{Développement 3D/VR Avancé}{Unity, C\#}
Conception d'une simulation immersive avec logique de combat avancée, IA ennemie et optimisation temps réel.

\project{Vision par Ordinateur}{PyTorch, Python}
Implémentation d’un CNN pour la reconnaissance de chiffres manuscrits et suivi AR.

\project{Cryptomonnaie Personnalisée}{Python, Web3, FastAPI}
Création d’un token blockchain et d’une interface Web ; démonstration de sécurité applicative.

% --- Languages & Interests ---
\section{Langues et Centres d'intérêt}

\textbf{Langues :} Français (C1), Anglais (C1), Arabe (natif).  

\textbf{Centres d’intérêt :}  
Développement de jeux (Unity / Unreal Engine).  
E-sport : joueur semi-pro (équipe INSA).



\end{document}
